\section{Introduction}
Social networks and blogs became very popular means of communication nowadays. Millions of users write their opinion about various aspects of everyday life. For this reason such services are very important data sources for the extraction and analysis of opinions. Having appeared and rapidly developed rather recently (first of all it's Twitter microblogs), these systems helped to discover more and more tools for the analysis of their content.

Text information can be divided into two main categories: facts and opinions. The facts are objective statements about some entities or events. Opinions are the subjective statements reflecting the  person's relation or perception of some event. The bulk of existing researches about natural language processing are concentrated on factual information collection and extraction. It's traditional information search, web search - in particular. But there haven't been enough researches in case of opinions handling. Though the accounting of people's opinions (subjective text information) is very useful not only for individuals, but also for the organizations. 

With the emergence and development of the Internet, and, as a result, with explosive growth of content created by its users new opportunities of information distribution and consumption were opened. There was an opportunity to publish feedbacks about products in online stores and to express the point of view concerning almost any things in various Internet-forums, blogs and social networks. As a result of all above-mentioned types of communication, the content volume created by users was increased. Now the consumers intending to make a purchase can find a huge number of the reviews and feedbacks estimating any goods in the network. When the organization wants to understand the relation of consumers concerning its products or services, traditionally, consultants are hired, surveys are conducted or focus groups are created by this organization. It could be avoided with tools for analysis of opinions in the network community. However the monitoring of sources with opinions is still a difficult task. There are a lot of various forums, blogs, etc. in the network, which contains a huge amount of information while the number of opinions in them on some specific question can be rather small. It's almost impossible to do some operations manually: to process such data arrays, to find and take necessary opinions from there, to distinguish their tonality, to generalize and lead result to a convenient form. Thus, the system of automatic collection and analysis of opinions is required.

